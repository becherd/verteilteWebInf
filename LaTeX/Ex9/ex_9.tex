\documentclass[a4paper, landscape]{article}

\usepackage[utf8]{inputenc}
\usepackage{enumitem}
\usepackage{tikz}
\usepackage{amsmath}
\usepackage{amssymb}

\usepackage{anysize}
\usepackage{multirow}
\usepackage{verbatim}
\usepackage{lscape}
\usepackage[landscape]{geometry}



\marginsize{2.0cm}{2cm}{2.5cm}{2.5cm}

\author{Gruppe 6}

\title{\textbf{VerteilteWebInf Hausaufgabe 9}}
\date{\today}




\begin{document}
\maketitle

\section*{Aufgabe 1}
%\begin{landscape}



\begin{tabular}{| l | l | l | l | l | l | p{6cm} |}
\hline
t & K & Status & Input & Entscheidung & Output & Kommentar \\
\hline
\hline


%%%%%%%%%
% STEP 1
%%%%%%%%%
1 & C & $UPDATE_{T1}(x)$ OK & - & - & $C: UPDATE_{T1}(x)$ OK & \\
\hline
\hline


%%%%%%%%%
% STEP 2
%%%%%%%%%
2 & \multirow{2}{*}{D} 
			&	$UPDATE_{T1}(x)$ OK 		& $C: UPDATE_{T1}(x)$ OK	& - &	 $C: UPDATE_{T1}(x)$ OK &\\
			& &															& 															   &   &	 $D: UPDATE_{T1}(x)$ OK &\\
\cline{2-7}
   & A & $UPDATE_{T2}(x)$ OK & - & - & $A: UPDATE_{T2}(x)$ OK & \\
\hline
\hline


%%%%%%%%%
% STEP 3
%%%%%%%%%
3 & \multirow{3}{*}{E} 
			&	$UPDATE_{T1}(x)$ OK 		& $C: UPDATE_{T1}(x)$ OK	& - &	 $C: UPDATE_{T1}(x)$ OK &\\
			& &															& $D: UPDATE_{T1}(x)$ OK &   &		 $D: UPDATE_{T1}(x)$ OK &\\
			& &															&																	& 	& 	 $E: UPDATE_{T1}(x)$ OK & \\

\cline{2-7}
& \multirow{2}{*}{B} 
			&	$UPDATE_{T2}(x)$ OK 		& $A: UPDATE_{T2}(x)$ OK	& - &	 $A: UPDATE_{T2}(x)$ OK &\\
			& &															& 															   &   &	 $B: UPDATE_{T2}(x)$ OK &\\
\cline{2-7}
 	& F & $UPDATE_{T3}(x)$ OK & - & - & $F: UPDATE_{T3}(x)$ OK & \\
\hline
\hline


%%%%%%%%%
% STEP 4
%%%%%%%%%
4 & \multirow{4}{*}{F} 
			&	$UPDATE_{T3}(x)$ OK 					 & $C: UPDATE_{T1}(x)$ OK	& - &	 $C: UPDATE_{T1}(x)$ OK &\\
			& &$UPDATE_{T1}(x)$ PASSIERE & $D: UPDATE_{T1}(x)$ OK &   &		 $D: UPDATE_{T1}(x)$ OK &\\
			& &																	   	& $E: UPDATE_{T1}(x)$ OK& 	& 	 $E: UPDATE_{T1}(x)$ OK & \\
			& &																	   	& 															& 	& 	 $F: UPDATE_{T1}(x)$ PASSIERE & \\
\cline{2-7}
& \multirow{2}{*}{C} 
			&   $UPDATE_{T1}(x)$ OK 		& $A: UPDATE_{T2}(x)$ OK	& - &	 -  &\\
			&& $UPDATE_{T2}(x)$ WARTE   &  $B: UPDATE_{T2}(x)$ OK  &   &	  & Warte, weil Antrag ($T_2$) mit niedriger priorisiertem Antrag ($T_1$) in Konflikt steht\\
\cline{2-7}
& \multirow{2}{*}{G} 
			&   $UPDATE_{T3}(x)$ OK 		& $F: UPDATE_{T3}(x)$ OK	& - &	$F: UPDATE_{T3}(x)$ OK &\\
			&& 															   & 															   &    & $G: UPDATE_{T3}(x)$ OK  & \\
\hline
\end{tabular}
\newpage
\begin{tabular}{| l | l | l | l | l | l | p{6cm} |}
\hline
t & K & Status & Input & Entscheidung & Output & Kommentar \\
\hline
\hline


%%%%%%%%%
% STEP 5
%%%%%%%%%
5 & \multirow{5}{*}{G} 
			&	$UPDATE_{T3}(x)$ OK 					 & $C: UPDATE_{T1}(x)$ OK	& - &	 $C: UPDATE_{T1}(x)$ OK &\\
			& &$UPDATE_{T1}(x)$ PASSIERE & $D: UPDATE_{T1}(x)$ OK &   &		 $D: UPDATE_{T1}(x)$ OK &\\
			& &																	   	& $E: UPDATE_{T1}(x)$ OK& 	& 	    $E: UPDATE_{T1}(x)$ OK & \\
			& &																	   	& $F: UPDATE_{T1}(x)$ PASSIERE& 	& 	 $F: UPDATE_{T1}(x)$ PASSIERE & \\
			& &																	   	&																					& 	& 	 $G: UPDATE_{T1}(x)$ PASSIERE & \\
\cline{2-7}
& \multirow{2}{*}{C} 
			&   $UPDATE_{T1}(x)$ OK 		& - 	& - &	 -  &\\
			&& $UPDATE_{T2}(x)$ WARTE   &   &   &	   &  \\
\cline{2-7}
& \multirow{2}{*}{A} 
			&    $UPDATE_{T2}(x)$ OK	      & $F: UPDATE_{T3}(x)$ OK	& - &	- &\\
			&&  $UPDATE_{T3}(x)$ WARTE  & 	$G: UPDATE_{T3}(x)$ OK  &    &   & \\
\hline
\hline


%%%%%%%%%
% STEP 6
%%%%%%%%%
6 & \multirow{6}{*}{A} 
			&    $UPDATE_{T2}(x)$ OK	      &  $C: UPDATE_{T1}(x)$ OK 	& - &	$C: UPDATE_{T1}(x)$ OK &\\
			&&  $UPDATE_{T3}(x)$ WARTE  & $D: UPDATE_{T1}(x)$ OK  &    &   $D: UPDATE_{T1}(x)$ OK & \\
			&&  $UPDATE_{T1}(x)$ PASSIERE & $E: UPDATE_{T1}(x)$ OK &    &  $E: UPDATE_{T1}(x)$ OK & \\
			& &																	   	& $F: UPDATE_{T1}(x)$ PASSIERE & & $F: UPDATE_{T1}(x)$ PASSIERE &\\
			& &																	   	& $G: UPDATE_{T1}(x)$ PASSIERE & & $G: UPDATE_{T1}(x)$ PASSIERE& \\
			& &																	   	&																					 & & $A: UPDATE_{T1}(x)$ PASSIERE& \\
\cline{2-7}
& \multirow{2}{*}{C} 
			&   $UPDATE_{T1}(x)$ OK 		& - 	& - &	 -  &\\
			&& $UPDATE_{T2}(x)$ WARTE   &   &   &	   &  \\
\hline
\hline


%%%%%%%%%
% STEP 7
%%%%%%%%%
7 & \multirow{6}{*}{B} 
			&    $UPDATE_{T2}(x)$ OK	      &  $C: UPDATE_{T1}(x)$ OK 	& ABORT $T_1$ & @ALL: ABORT $T_1$ & ABORT, weil $T_1$ das geforderte\\
			&&  $UPDATE_{T3}(x)$ WARTE  & $D: UPDATE_{T1}(x)$ OK  &    &   & Quorum (4 mal OK) nicht mehr \\
			&&  $UPDATE_{T1}(x)$ PASSIERE & $E: UPDATE_{T1}(x)$ OK &    &  & erreichen kann \\
			& &																	   	& $F: UPDATE_{T1}(x)$ PASSIERE & & &\\
			& &																	   	& $G: UPDATE_{T1}(x)$ PASSIERE & & & \\
			& &																	   	& $A: UPDATE_{T1}(x)$ PASSIERE & & & \\
\cline{2-7}
& \multirow{2}{*}{C} 
			&   $UPDATE_{T1}(x)$ OK 		& - 	& - &	 -  &\\
			&& $UPDATE_{T2}(x)$ WARTE   &   &   &	   &  \\
\cline{2-7}
& \multirow{2}{*}{A} 
			&   $UPDATE_{T2}(x)$ OK 		& - 	& - &	 -  &\\
			&& $UPDATE_{T3}(x)$ WARTE   &   &   &	   &  \\
			&&  $UPDATE_{T1}(x)$ PASSIERE &  &  &   & \\
\hline
\end{tabular}
\newpage
\begin{tabular}{| l | l | l | l | l | l | p{6cm} |}
\hline
t & K & Status & Input & Entscheidung & Output & Kommentar \\
\hline
\hline

%%%%%%%%%
% STEP 8
%%%%%%%%%
8 & \multirow{2}{*}{A} 
			&   $UPDATE_{T2}(x)$ OK 		&   ABORT $T_1$ 	& - &	 -  &\\
			&& $UPDATE_{T3}(x)$ WARTE   &   &   &	   &  \\
\cline{2-7}
& \multirow{3}{*}{C} 
			& $UPDATE_{T2}(x)$ OK   &  ABORT $T_1$  & -  &	 $A: UPDATE_{T2}(x)$ OK  &  \\
			&& 																		  & 									 &  &	 $B: UPDATE_{T2}(x)$ OK  &  \\
			&& 																		  & 									 &  &	 $C: UPDATE_{T2}(x)$ OK  &  \\
\cline{2-7}
& D & & ABORT $T_1$& - & - &\\
\cline{2-7}
& E & & ABORT $T_1$& - & - &\\
\cline{2-7}
& F &	$UPDATE_{T3}(x)$ OK 					 & ABORT $T_1$	& - & -&\\
\cline{2-7}
& G &	$UPDATE_{T3}(x)$ OK 					 & ABORT $T_1$	& - & -&\\
\hline
\hline

%%%%%%%%%
% STEP 9
%%%%%%%%%
9 & \multirow{2}{*}{D} 
			&   $UPDATE_{T2}(x)$ OK & $A: UPDATE_{T2}(x)$ OK 		  	& COMMIT $T_2$ &	@ALL: COMMIT $T_2$  &\\
			&& 																& $B: UPDATE_{T2}(x)$ OK      &   &	   &  \\
			&&																	& $C: UPDATE_{T2}(x)$ OK      &   &	   &  \\
\cline{2-7}
& \multirow{2}{*}{A} 
			&   $UPDATE_{T2}(x)$ OK 		&   -	& - &	 -  &\\
			&& $UPDATE_{T3}(x)$ WARTE   &   &   &	   &  \\
\hline
\hline


%%%%%%%%%
% STEP 10
%%%%%%%%%
10 & A & $UPDATE_{T3}(x)$ ABGELEHNT   &  COMMIT $T_2$ &  ABORT $T_3$ &	 @ALL:ABORT $T_3$ &  Durch das COMMIT wurden die Zeitstempel aktualisiert. Dadurch ist sind die Zeitstempel von $F: UPDATE_{T3}(x)$ und $G: UPDATE_{T3}(x)$, die Knoten A in Zeitschritt 5 von Knoten $G$ empfangen hat, veraltet. Deshalb wird ein ABORT ausgelöst.\\
\hline

					
\end{tabular}
%\end{landscape}

\section*{Aufgabe 2}
\begin{enumerate}[label=\alph*)]
\item 
\begin{tabular}{| l | l | l | l | l | l | l | l |}
\hline 
i & 3 & 14 & 26 & 41 & 56\\
\hline
\hline
0 i+1 & P11 & P26 & P41 & P51 & P61\\
1 i+2 & P11 & P26 & P41 & P51 & P61\\
2 i+4 & P11 & P26 & P41 & P51 & P61\\
3 i+8 & P11 & P26 & P41 & P51 & P3\\
4 i+16 & P26 & P41 & P51 & P61 & P11\\
5 i+32 & P41 & P51 & P61 & P11 & P26\\
\hline
\end{tabular}

\item 
Bei jedem Schritt der Suche kann der Suchbereich weiter eingegrenzt werden, es wird also immer mindestens ein genauerer Eintrag der Fingertabelle als der vorherige Eintrag betrachtet. \\
Bei einem Kreis mit maximal m verschiedenen möglichen Werten ist der Abstand zwischen dem aktuellen Knoten $ A_i $, an dem im i-ten Schritt gesucht wird, und dem gesuchten Knoten x maximal: $ \vert A_i-x \vert \leq \lfloor \frac{m}{2^i} \rfloor $. Dies ergibt sich daraus, dass der mögliche Bereich immer mindestens halbiert wird, da ansonsten aufeinander folgende Schritte widersprüchliche Ergebnisse liefern. \\
Beendet werden kann der Algorithmus spätestens, wenn der Abstand zwischen dem aktuellen und dem gesuchten Knoten 1 ist, da dann der nächste Knoten des gesuchte ist.Dann gilt: $ \vert A_i-x \vert = 1 = \lfloor \frac{m}{2^i} \rfloor \Leftrightarrow m = 2^i \Leftrightarrow i = log_2 m $\\
Der Aufwand für die Suche in diese Netzwerk ist somit maximal logarithmisch.

\item 
\begin{itemize}
\item Suche K90 ($ \rightarrow 90-64 = 26 $) ausgehend von P3: P3 - P26 (gefunden!) 
\item Suche K8 ausgehend von P11: P11 - P51 - P3 - P11
\item Suche K2 ausgehend von P11: P11 - P51 - P61 - P3
\item Suche K258 ($ \rightarrow 258-64*4 = 2 $) ausgehend von P61: P61 - P3 (gefunden!) 
\end{itemize}

\item Hinzufügen von P33
Suchen der Stelle, wo P33 eingefügt werden muss: vor P41\\
P41 informiert P33 über seinen Vorgänger und P33 meldet sich bei P26 an. \\
 Dann müssen die anderen Peers über den hinzugefügten Peer informiert werden, damit sie ihre Fingertabellen aktualisieren können. Daraufhin kann P33 die ihm zugeteilten Daten von P41 (von K27 bis K33 (*64)) übernehmen.

\item Fingertabellen nach Hinzufügen von P33:
\begin{tabular}{| l | l | l | l | l | l | l | l | l |}
\hline 
i & 3 & 11 & 14 & 26 & 41 & 51 & 56 & 61\\
\hline
\hline
0 i+1 & P11 & P14 & P26 & \underline{P33} & P51 & P56 & P61 & P3\\
1 i+2 & P11 & P14 & P26 & \underline{P33} & P51 & P56 & P61 & P3\\
2 i+4 & P11 & P26 & P26 & \underline{P33} & P51 & P56 & P61 & P3\\
3 i+8 & P11 & P26 & P26 & P41 & P51 & P61 & P3 & P11\\
4 i+16 & P26 & \underline{P33} & \underline{P33} & P51 & P61 & P3 & P11 & P14\\
5 i+32 & P41 & P51 & P51 & P61 & P11 & P26 & P26 & \underline{P33}\\

\hline
\end{tabular}

\end{enumerate}

\end{document}


