\documentclass[a4paper]{scrartcl}

\usepackage[utf8]{inputenc}
\usepackage{enumitem}
\usepackage{tikz}
\usepackage{amsmath}
\usepackage{amssymb}

\usepackage{anysize}
\usepackage{multirow}
\usepackage{verbatim}
\usepackage{lscape}
\usepackage{graphicx}

\marginsize{2.0cm}{2cm}{2.5cm}{2.5cm}

\author{Gruppe 6}

\title{\textbf{VerteilteWebInf Hausaufgabe 13}}
\date{\today}


\begin{document}
\maketitle

\section*{Aufgabe 1}
Wenn der Zufalls-U-Bahn-Fahrer in Garching Forschungszentrum beginnt, benötigt 1401 Schritte um das Ergebnis auf 5 Nachkommastellen exakt zu berechnen.\\
Startet der U-Bahn-Fahrer in Hauptbahnhof, Sendlinger Tor, Scheidplatz, Innsbrucker Ring oder Odeonsplatz, werden nur 802 Schritte für dieselbe Genauigkeit benötigt.

\section*{Aufgabe 2}

\end{document}


