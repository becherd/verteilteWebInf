\documentclass[a4paper]{article}

\usepackage[utf8]{inputenc}
\usepackage{enumitem}


\usepackage{verbatim}

\author{Gruppe 6}

\title{\textbf{VerteilteWebInf Hausaufgabe 4}}
\date{\today}




\begin{document}
\maketitle


\section*{Aufgabe 1}

\subsubsection*{Globale Prädikate:}
\texttt{auto(A)}: Auto \texttt{A}\\
\texttt{marke(A,M)}: Marke \texttt{M} \\
\texttt{modell(A,M,T)}: Modell \texttt{T} der Marke \texttt{M}\\
\texttt{maxPreis(A,P)}: Maximaler Preis \texttt{P} \\
\texttt{erstzulassung(A,E)}: Erstzulassung \texttt{E} \\
\texttt{kilometerstand(A,Km)}: Kilometerstand \texttt{Km}\\
\texttt{kraftstoff(A,K)}: Kraftstoffart \texttt{K}\\
\texttt{ort(A,O)}: Ortsname \texttt{O}\\
\texttt{plz(A,Plz)}: Postleitzahl \texttt{Plz}\\
\texttt{land(A,L)}: Land \texttt{L}\\
\texttt{umkreisO(A,O,U)}: Umkreis \texttt{U} um Ort mit Name \texttt{O}\\
\texttt{umkreisPlz(A,Plz,U)}: Umkreis \texttt{U} um Ort mit PLZ \texttt{Plz} \\
\texttt{Garantie(A,G)}: Garantie \texttt{G} (\emph{true} oder \emph{false})\\


\subsection*{Information Manifold}


\subsubsection*{Datenquellen:}
\begin{verbatim}
mobile(A,M,T,P,E,Km,K,Plz,L,U) :- auto(A) & marke(A,M) & modell(A,M,T) 
& maxPreis(A,P) & erstzulassung(A,E) & kilometerstand(A,Km) 
& kraftstoff(A,K) & plz(A,Plz) & land(A,L) & umkreisPlz(A,Plz,U).


autoScout(A,M,T,P,E,Km,K,Plz,O,U,G) :- auto(A) & marke(A,M) & modell(A,M,T) 
& maxPreis(A,P) & erstzulassung(A,E) & kilometerstand(A,Km) & plz(A,Plz)
& kraftstoff(A,K) & ort(A,O) & umkreisO(A,O,U) & garantie(A,G).

autoScout(A,M,T,P,E,Km,K,Plz,O,U,G) :- auto(A) & marke(A,M) & modell(A,M,T) 
& maxPreis(A,P) & erstzulassung(A,E) & kilometerstand(A,Km) & ort(A,O)
& kraftstoff(A,K) & plz(A,Plz) & umkreisPlz(A,Plz,U) & garantie(A,G).
\end{verbatim}
Die Anfragen werden gegen die globalen Prädikate gestellt.

\subsection*{Tsimmis}
\subsubsection*{Datenquellen:}
\begin{verbatim}
mobile(A,M,T,P,E,Km,K,Plz,L,U) :- auto(A) & marke(A,M) & modell(A,M,T) 
& maxPreis(A,P) & erstzulassung(A,E) & kilometerstand(A,Km) 
& kraftstoff(A,K) & plz(A,Plz) & land(A,L) & umkreisPlz(A,Plz,U).


autoScout(A,M,T,P,E,Km,K,Plz,O,U,G) :- auto(A) & marke(A,M) & modell(A,M,T) 
& maxPreis(A,P) & erstzulassung(A,E) & kilometerstand(A,Km) & plz(A,Plz)
& kraftstoff(A,K) & ort(A,O) & umkreisO(A,O,U) & garantie(A,G).

autoScout(A,M,T,P,E,Km,K,Plz,O,U,G) :- auto(A) & marke(A,M) & modell(A,M,T) 
& maxPreis(A,P) & erstzulassung(A,E) & kilometerstand(A,Km) & ort(A,O)
& kraftstoff(A,K) & plz(A,Plz) & umkreisPlz(A,Plz,U) & garantie(A,G).
\end{verbatim}

\subsubsection*{Mediatoren:}
\begin{verbatim}
kennzahlenAuto(A,M,T,E,Km,K) :- autoScout(A,M,T,P,E,Km,K,Plz,O,U,G) &
 mobile(A,M,T,P,E,Km,K,Plz,L,U).

kaufInfos(A,P,G) :- autoScout(A,M,T,P,E,Km,K,Plz,O,U,G) &
 mobile(A,M,T,P,E,Km,K,Plz,L,U).

ortInfos(A,O,Plz,L,UOrt,UPlz) :- autoScout(A,M,T,P,E,Km,K,Plz,O,U,G) &
 mobile(A,M,T,P,E,Km,K,Plz,L,U).
\end{verbatim}
Bei Tsimmis werden die Anfragen gegen die Mediatoren gestellt.

\section*{Aufgabe 2}
\textbf{Prädikat $ A_1 $}:\\
$ Q' = \{ A_1 \}, M(Q') = \{ A_1^+, A_1^-\} $\\
$ F(Q') = \{ AbtNr\leq 250, AbtNr>250\} $\\
ja, es gibt eine wesentliche Verbesserung und noch keine unnötige Partitionierung, also $ Q = \{ A_1 \} $\\
\\
\textbf{Prädikat $ A_2 $}:\\
$ Q' = \{ A_1, A_2 \}, M(Q') = \{ A_1^+ A_2^+, A_1^+ A_2^-, A_1^- A_2^+, A_1^- A_2^-\} $\\
kurz: $ M(Q') = \{ A_1^+, A_2^+, A_1^- A_2^-\} $\\
$ F(Q') = \{ AbtNr\leq 250, AbtNr>250 \wedge AbtNr \leq 400, AbtNr > 400 \} $\\
ja, es gibt eine wesentliche Verbesserung, also $ Q = \{ A_1, A_2 \} $ \\
\textit{unnötige Partitionierung} bzgl. $ A_1 $?: mit $ A_1 $ in Q wesentliche Verbesserung, also keine unnötige Partitionierung\\
\\
\textbf{Prädikat $ A_3 $}:\\
$ Q' = \{ A_1, A_2, A_3 \}, \\
M(Q') = \{ A_1^+ A_2^+ A_3^+, A_1^+ A_2^- A_3^+, A_1^- A_2^+ A_3^+, A_1^- A_2^- A_3^+, A_1^+ A_2^+ A_3^-, A_1^+ A_2^- A_3^-, A_1^- A_2^+ A_3^-, A_1^- A_2^- A_3^-\} $\\
kurz: $ M(Q') = \{ A_1^+, A_2^+, A_3^+ \} $\\
$ F(Q') = \{ AbtNr\leq 250, AbtNr>250 \wedge AbtNr \leq 400, AbtNr > 400 \} $\\
nein, es gibt keine wesentliche Verbesserung, also bleibt $ Q = \{ A_1, A_2 \} $ \\
\\
\textbf{Prädikat $ A_4 $}:\\
$ Q' = \{ A_1, A_2, A_4 \}, \\
M(Q') = \{ A_1^+ A_2^+ A_4^+, A_1^+ A_2^- A_4^+, A_1^- A_2^+ A_4^+, A_1^- A_2^- A_4^+, A_1^+ A_2^+ A_4^-, A_1^+ A_2^- A_4^-, A_1^- A_2^+ A_4^-, A_1^- A_2^- A_4^-\} $\\
kurz: $ M(Q') = \{ A_1^+ A_4^+, A_1^+ A_4^-, A_2^+, A_1^- A_2^-\} $\\
$ F(Q') = \{ AbtNr\leq 150, AbtNr >150 \wedge AbtNr \leq 250, AbtNr>250 \wedge AbtNr \leq 400, AbtNr > 400 \} $\\
ja, es gibt eine wesentliche Verbesserung, also $ Q = \{ A_1, A_2, A_4 \} $ \\
\textit{unnötige Partitionierung} bzgl. $ A_1 $?: \\
$ Q' = \{A_2, A_4) \}, M(Q') = \{ A_2^+ A_4^+, A_2^+ A_4^-, A_2^- A_4^+, A_2^- A_4^- \} $\\
$ F(Q') = \{ AbtNr\leq 150, (AbtNr >150 \wedge AbtNr \leq 250) \vee (AbtNr > 400), AbtNr>250 \wedge AbtNr \leq 400 \} $
mit $ A_1 $ in Q wesentliche Verbesserung, also keine unnötige Partitionierung und $ Q = \{ A_1, A_2, A_4 \} $\\
\textit{unnötige Partitionierung} bzgl. $ A_2 $?: \\
$ Q' = \{A_1, A_4 \}, M(Q') = \{ A_1^+ A_4^+, A_1^+ A_4^-, A_1^-\} $\\
$ F(Q') = \{ AbtNr\leq 150, AbtNr >150 \wedge AbtNr \leq 250, AbtNr >250 \} $
mit $ A_2 $ in Q wesentliche Verbesserung, also keine unnötige Partitionierung und $ Q = \{ A_1, A_2, A_4 \} $\\
\\
%Prädikat A_5
\textbf{Prädikat $ A_5 $}:\\
$ Q' = \{ A_1, A_2, A_4, A_5 \} $\\
relevante Minterme: $ M(Q') = \{ A_4^+, A_1^+ A_5^+, A_2^+ A_5^+, A_2^+ A_5^-, A_1^- A_2^- \} $\\
$ F(Q') = \{ AbtNr\leq 150, AbtNr >150 \wedge AbtNr \leq 250, AbtNr>250 \wedge AbtNr \leq 299, AbtNr >299 \wedge AbtNr \leq 400, AbtNr > 400 \} $\\
ja, es gibt eine wesentliche Verbesserung, also $ Q = \{ A_1, A_2, A_4, A_5 \} $ \\
\textit{unnötige Partitionierung} bzgl. $ A_1 $?: \\
$ Q' = \{A_2, A_4, A_5) \}, M(Q') = \{ A_4^+, A_2^- A_5^+, A_2^+ A_5^+, A_2^+ A_5^-, A_2^- A_4^- A_5^- \} $\\
$ F(Q') = \{ AbtNr\leq 150, AbtNr >150 \wedge AbtNr \leq 250, AbtNr>250 \wedge AbtNr \leq 299, AbtNr >299 \wedge AbtNr \leq 400, AbtNr > 400 \} $
mit $ A_1 $ in Q keine wesentliche Verbesserung, also unnötige Partitionierung und $ Q = \{ A_2, A_4, A_5 \} $\\
\textit{unnötige Partitionierung} bzgl. $ A_2 $?: \\
$ Q' = \{A_4, A_5 \}, M(Q') = \{ A_4^+ A_5^+, A_4^- A_5^+, A_4^+ A_5^-, A_4^- A_5^- \} $\\
$ F(Q') = \{ AbtNr\leq 150, AbtNr >150 \wedge AbtNr \leq 299, AbtNr > 299\} $
mit $ A_2 $ in Q wesentliche Verbesserung, also keine unnötige Partitionierung und $ Q = \{ A_2, A_4, A_5 \} $\\
\textit{unnötige Partitionierung} bzgl. $ A_4 $?: \\
$ Q' = \{A_2, A_5 \}, M(Q') = \{ A_2^+ A_5^+, A_2^- A_5^+, A_2^+ A_5^-, A_2^- A_5^- \} $\\
$ F(Q') = \{ AbtNr\leq 150 \vee AbtNr > 400, AbtNr >150 \wedge AbtNr \leq 250, AbtNr>250 \wedge AbtNr \leq 299, AbtNr >299 \wedge AbtNr \leq 400\} $
mit $ A_2 $ in Q wesentliche Verbesserung, also keine unnötige Partitionierung und $ Q = \{ A_2, A_4, A_5 \} $\\
\\
%Prädikat A6
\textbf{Prädikat $ A_6 $}:\\
$ Q' = \{ A_2, A_4, A_5, A_6 \} $\\
relevante Minterme: $ M(Q') = \{ A_4^+, A_2^- A_5^+, A_2^+ A_5^+, A_2^+ A_5^-, A_2^- A_4^- A_5^- \} $\\
$ F(Q') = \{ AbtNr\leq 150, AbtNr >150 \wedge AbtNr \leq 250, AbtNr>250 \wedge AbtNr \leq 299, AbtNr >299 \wedge AbtNr \leq 400, AbtNr > 400 \} $\\
nein, es gibt keine wesentliche Verbesserung, also bleibt $ Q = \{ A_2, A_4, A_5 \} $ \\
\\
\textbf{resultierende Partitionierung}:\\
$ F(Q') = \{ AbtNr\leq 150, AbtNr >150 \wedge AbtNr \leq 250, AbtNr>250 \wedge AbtNr \leq 299, AbtNr >299 \wedge AbtNr \leq 400, AbtNr > 400 \} $\\
Da AbtNr ein Primärschlüssel ist wird davon ausgegangen, dass NULL-Werte bei der Partitionierung nicht berücksichtigt werden müssen. 





\end{document}


