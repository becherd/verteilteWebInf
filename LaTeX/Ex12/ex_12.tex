\documentclass[a4paper]{scrartcl}

\usepackage[utf8]{inputenc}
\usepackage{enumitem}
\usepackage{tikz}
\usepackage{amsmath}
\usepackage{amssymb}

\usepackage{anysize}
\usepackage{multirow}
\usepackage{verbatim}
\usepackage{lscape}
\usepackage{graphicx}

\marginsize{2.0cm}{2cm}{2.5cm}{2.5cm}

\author{Gruppe 6}

\title{\textbf{VerteilteWebInf Hausaufgabe 12}}
\date{\today}


\begin{document}
\maketitle

\section*{Aufgabe 1}
\begin{itemize}
\item Neu geschribene Engine in C++ und Java. Die meisten anderen Engines basieren auf Postgres.
\item Benutzt Hadoop Standard Komponenten.
\item Unterstützt Industriestandards wie  OBDC oder JBDC, sowie Authentifizierung über Kerberos oder LDAP.
\item Spalten können mit "PARTITIONED" partitioniert werden und mit LOCATION kann der hdfs path angegeben werden.
\item Als Dateiformat wird RCFile, Avro (binäres Format) und Parquet unterstützt. Weiters können Partitionen innerhalb einer Tabelle verschiedene Formate haben.
\item Standart SQL SELECT Syntax (SQL-92, SQL-2003)

\end{itemize}

\section*{Aufgabe 2}
\begin{verbatim}
SELECT ?actor WHERE
{ ?actor <actedIn> ?movie.
  ?actor <directed> ?movie2.
}
\end{verbatim}




\end{document}


