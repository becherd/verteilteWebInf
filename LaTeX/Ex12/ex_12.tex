\documentclass[a4paper]{scrartcl}

\usepackage[utf8]{inputenc}
\usepackage{enumitem}
\usepackage{tikz}
\usepackage{amsmath}
\usepackage{amssymb}

\usepackage{anysize}
\usepackage{multirow}
\usepackage{verbatim}
\usepackage{lscape}
\usepackage{graphicx}

\marginsize{2.0cm}{2cm}{2.5cm}{2.5cm}

\author{Gruppe 6}

\title{\textbf{VerteilteWebInf Hausaufgabe 12}}
\date{\today}


\begin{document}
\maketitle

\section*{Aufgabe 1}
Impala: open-source, massively parallel processing (MPP) SQL-Query-Engine
\begin{itemize}
\item geringe Latenz: verteilte Architektur auf Daemon-Prozessen aufgebaut, die für Anfrageausführung zuständig sind
\item hohe Nebenläufigkeit
\item komplett neu geschriebenes System in C++/Java
\item HDFS als darunterliegender Speichermanager optimal
\item sehr schnell: \\
single-user queries: bis zu 13-mal schneller als alternative Systeme; durchschnittlich 6.7-mal schneller\\
multi-user queries: durchschnittlich bis zu 18xschneller

\item läuft auf hunderten Rechnern in Hadoop-Cluster; unabhängig von darunterliegender Speicherarchitektur
\item drei Services: Query Planner, Query Coordinator, Query Executor (alle Daemons können in allen Rollen auftreten)
\item Datenlokalität: jeder Knoten akzeptiert Anfragen und führt sie aus; Synchronisierung parallel (keine RPCs; Push der wichtigen Informationen an alle Interessenten (Subscribers)
\item unterstützt Standards: z.B. JDBC/ODBC, Authentikation über Kerberos/LDAP, SQL
\end{itemize}

\textbf{Frontend (Java)}
\begin{itemize}
\item SQL $ \rightarrow $ Query Plan: Parsing, semantische Analyse, Optimierung
\item Query Planner: Ein-Rechner-System-Planung $ \rightarrow $ Parallelisierung (maximale Lokalität, minimaler Datentransfer): Broadcast- oder partitionierter Join; lokale Prä-Aggregation + Merge
\end{itemize}

\textbf{Backend (C++)}
\begin{itemize}
\item Codegenerierung zur Laufzeit (mit LLVM; für Code, der oft ausgeführt wird), Pipelining
\item I/O mittels HDFS-Feature short-circuit local reads (lesen mit fast komplett Bandbreite)
\item Speicherformate: meist Apache Parquet (hohe Kompression, hohe Effizienz): Encoding zur Laufzeit, Statistiken zur Optimierung
\end{itemize}

\textbf{Resource/Workload Management}
\begin{itemize}
\item Apache YARN: zentraler Ressourcen-Manager, volles Wissen über Lastverteilung
\item eigene Implementierung: verteilte Kontrolle + Llama (für caching, Scheduling) + YARN
\item Llama/YARN: alle Ressourcenanfragen an Llama; falls Ressourcen in Cache verfügbar, dann Zuteilung; ansonsten Weiterleitung zu YARN
\end{itemize}


\section*{Aufgabe 2}
\begin{verbatim}
SELECT ?actor WHERE
{ ?actor <actedIn> ?movie.
  ?actor <directed> ?movie2.
}
\end{verbatim}
bzw.
\begin{verbatim}
PREFIX dbpedia-owl: <http://dbpedia.org/ontology/>
PREFIX dbpedia: <http://dbpedia.org/resource/>
PREFIX dbpprop: <http://dbpedia.org/property/>

SELECT ?actor ?name
WHERE {
?actor dbpedia-owl:occupation dbpedia:Actor.
?actor dbpedia-owl:occupation dbpedia:Film_director.
OPTIONAL {?actor dbpprop:name ?name.}
} ORDER BY ?actor
\end{verbatim}



\end{document}


