\documentclass[a4paper]{article}

\usepackage[utf8]{inputenc}
\usepackage{enumitem}
\usepackage{tikz}
\usepackage{amsmath}
\usepackage{amssymb}

\usepackage{anysize}

\usepackage{verbatim}


\marginsize{2.0cm}{2cm}{2.5cm}{2.5cm}

\author{Gruppe 6}

\title{\textbf{VerteilteWebInf Hausaufgabe 6}}
\date{\today}




\begin{document}
\maketitle


\section*{Aufgabe 1}
False-Drop-Abschätzung beim Bloom-Filter (für Join zwischen R und S): \\
Abbildung auf einen Hash-Filter der Länge b (0, ..., b-1) mit k Hash-Funktionen:
\begin{itemize}
\item Wahrscheinlichkeit, dass kein $ r \in R $ ein bestimmtes Bit setzt: $ (1-1/b)^{k* \vert R \vert} ((1-1/b)^b)^{k*\vert R \vert / b} \approx e^{-k*\vert R \vert / b} $ (für große b, Definition der e-Funktion)\\
Wahrscheinlichkeit, dass ein $ r \in R $ das Bit gesetzt hat: $ (1- (1-1/b)^{k*\vert R \vert})^k \approx (1-e^{-k*\vert R \vert / b})^k $
\item Wahrscheinlichkeit, dass ein bestimmtes $ s \in S $ ausgewählt wird: $ (1- (1-1/b)^{k*\vert R \vert})^k \approx (1-e^{-k*\vert R \vert / b})^k $ = Wahrscheinlichkeit für ein false positive
\end{itemize}









\end{document}


