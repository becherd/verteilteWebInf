\documentclass[a4paper]{article}

\usepackage[utf8]{inputenc}
\usepackage{enumitem}
\usepackage{tikz}
\usepackage{amsmath}
\usepackage{amssymb}

\usepackage{anysize}

\usepackage{verbatim}


\marginsize{2.0cm}{2cm}{2.5cm}{2.5cm}

\author{Gruppe 6}

\title{\textbf{VerteilteWebInf Hausaufgabe 8}}
\date{\today}




\begin{document}
\maketitle


\section*{Aufgabe 3}
\begin{enumerate}[label=\alph*)]
\item Obermarck-Algorithmus:\\
Knoten A: 
\begin{itemize}
\item $EX\rightarrow T9\rightarrow T1 \rightarrow EX$
\item $EX\rightarrow T8\rightarrow T3 \rightarrow T1 \rightarrow EX$
\item $EX\rightarrow T5\rightarrow T3 \rightarrow T1 \rightarrow EX$
\end{itemize}
Knoten B: 
\begin{itemize}
\item $EX\rightarrow T1\rightarrow T5 \rightarrow EX$
\end{itemize}
Knoten C: 
\begin{itemize}
\item $EX\rightarrow T4\rightarrow T7 \rightarrow T2 \rightarrow EX$
\item $EX\rightarrow  T5 \rightarrow EX$
\end{itemize}
Knoten A schickt nun alle Pfade seines Wartegraphen an B ($T1<T9$, $T1<T8$, $T1<T5$, $T1$ hat jeweils Sub-Transaktion in B), außerdem schickt Knoten C seinen ersten Pfan an B ($T2<T4$, $T2$ hat Sub-Transaktion in B). Zusätzliche Kanten in B:
\begin{itemize}
\item $EX\rightarrow T9\rightarrow T1 \rightarrow EX$
\item $EX\rightarrow T8\rightarrow T3 \rightarrow T1 \rightarrow EX$
\item $EX\rightarrow T5\rightarrow T3 \rightarrow T1 \rightarrow EX$
\item $EX\rightarrow T4\rightarrow T7 \rightarrow T2 \rightarrow EX$
\end{itemize}
Im Knoten B ist nun ein Zyklus ohne $EX$ entstanden, nämlich $T5\rightarrow T3\rightarrow T1 \rightarrow T5$, d.h. es wurde ein Deadlock erkannt. Nun muss ein Opfer in diesem Zyklus ausgewählt und zurückgesetzt werden. Wähle z.B. $T5$ als Opfer aus. Weitere Zyklen sind nicht vorhanden, sodass das System anschließend Deadlock-frei ist.
\end{enumerate}
\end{document}


