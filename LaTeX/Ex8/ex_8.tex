\documentclass[a4paper]{article}

\usepackage[utf8]{inputenc}
\usepackage{enumitem}
\usepackage{tikz}
\usepackage{amsmath}
\usepackage{amssymb}

\usepackage{anysize}

\usepackage{verbatim}


\marginsize{2.0cm}{2cm}{2.5cm}{2.5cm}

\author{Gruppe 6}

\title{\textbf{VerteilteWebInf Hausaufgabe 8}}
\date{\today}




\begin{document}
\maketitle

\section*{Aufgabe 1}
CA und CP sind sehr ähnlich:
\begin{itemize}
\item CA bedeutet, dass das System verfügbar ist, solange das Netzwerk intakt ist und keine Nachrichten verloren gehen, d.h. solange es keine \emph{network partition} gibt.
\item CP bedeutet, dass das System nicht verfügbar ist, wenn es eine \emph{network partition} gibt.
\end{itemize}
Im Prinzip sind CA und CP also gleich.


\section*{Aufgabe 3}
\begin{enumerate}[label=\alph*)]
\item Obermarck-Algorithmus:\\
Knoten A: 
\begin{itemize}
\item $EX\rightarrow T9\rightarrow T1 \rightarrow EX$
\item $EX\rightarrow T8\rightarrow T3 \rightarrow T1 \rightarrow EX$
\item $EX\rightarrow T5\rightarrow T3 \rightarrow T1 \rightarrow EX$
\end{itemize}
Knoten B: 
\begin{itemize}
\item $EX\rightarrow T1\rightarrow T5 \rightarrow EX$
\end{itemize}
Knoten C: 
\begin{itemize}
\item $EX\rightarrow T4\rightarrow T7 \rightarrow T2 \rightarrow EX$
\item $EX\rightarrow  T5 \rightarrow EX$
\end{itemize}
Knoten A schickt nun alle Pfade seines Wartegraphen an B ($TransID(T1)<TransID(T9)$, $TransID(T1)<TransID(T8)$,  $TransID(T1)<TransID(T5)$, $T1$ hat jeweils Sub-Transaktion in B), außerdem schickt Knoten C seinen ersten Pfan an B ($TransID(T2)<TransID(T4)$, $T2$ hat Sub-Transaktion in B). Zusätzliche Kanten in B:
\begin{itemize}
\item $EX\rightarrow T9\rightarrow T1 \rightarrow EX$
\item $EX\rightarrow T8\rightarrow T3 \rightarrow T1 \rightarrow EX$
\item $EX\rightarrow T5\rightarrow T3 \rightarrow T1 \rightarrow EX$
\item $EX\rightarrow T4\rightarrow T7 \rightarrow T2 \rightarrow EX$
\end{itemize}
Im Knoten B ist nun ein Zyklus ohne $EX$ entstanden, nämlich $T5\rightarrow T3\rightarrow T1 \rightarrow T5$, d.h. es wurde ein Deadlock erkannt. Nun muss ein Opfer in diesem Zyklus ausgewählt und zurückgesetzt werden. Wähle z.B. $T5$ als Opfer aus. Weitere Zyklen sind nicht vorhanden, sodass das System anschließend Deadlock-frei ist.

\item Angenommen, es gibt einen Deadlock $T_i\rightarrow...\rightarrow T_k \rightarrow ...\rightarrow T_l \rightarrow ... \rightarrow T_i$.\\
Dann muss es in mindestens einem Knoten (z.B. $K_1$) den Pfad $EX\rightarrow T_k \rightarrow ... \rightarrow T_l \rightarrow EX$ und in mindestens einem anderen Knoten (z.B. $K_2$) den Pfad $EX\rightarrow T_l \rightarrow ... \rightarrow T_k \rightarrow EX$ geben. Dann ist es ausreichend, wenn ein Knoten diese beiden Pfade kennt, um den Deadlock festzustellen. Es reicht also, wenn Knoten $K_2$ seinen Pfad an $K_1$ schickt, also wenn nur Pfade  $EX\rightarrow T_l \rightarrow ... \rightarrow T_k \rightarrow EX$ verschickt werden, bei denen $TransID(T_l)>TransID(T_k)$ gilt.
\end{enumerate}
\end{document}


