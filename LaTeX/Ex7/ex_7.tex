\documentclass[a4paper]{article}

\usepackage[utf8]{inputenc}
\usepackage{enumitem}
\usepackage{tikz}
\usepackage{amsmath}
\usepackage{amssymb}

\usepackage{anysize}

\usepackage{verbatim}


\marginsize{2.0cm}{2cm}{2.5cm}{2.5cm}

\author{Gruppe 6}

\title{\textbf{VerteilteWebInf Hausaufgabe 7}}
\date{\today}




\begin{document}
\maketitle


\section*{Aufgabe 1}
\begin{enumerate}[label=\alph*)]
\item Annahme: Es wären zwei Joinpartner $ r \in R $ und $ s \in S $ in unterschiedlichen Partitionen $ \mathcal{R}_i $ und $ \mathcal{R}_j $.\\
Dann gilt $ h(r.A) mod $  $p \neq h(s.B) mod $  $ p $ bei gleicher Hashfunktion h, also gilt $ r.A \neq s.B $, was ein Widerspruch dazu ist, dass r und s Joinpartner sind. 
\item unter der Annahme, dass jedem Knoten genau eine Partition zugeteilt wird muss Knoten i \\
$ \dfrac{p-1}{p} \mid \mathcal{R}_i \mid + \dfrac{p-1}{p} \mid \mathcal{S}_i \mid $ Tupel verschicken.\\
Zu addieren sind noch die Ergebnistupel aus dem Join von $ \mathcal{R}_{j,i} $ und $ \mathcal{R}_{j,i} \forall j$, die gegebenenfalls an eine andere Station gesendet werden müssen.
\item Bei einem Join mit Broadcast verschickt jeder Knoten $ \mid \mathcal{R} \mid $ Tupel.\\
Zu addieren sind noch die Ergebnistupel aus dem Join von $ \mathcal{R} $ und $ \mathcal{S}_{i} $, die gegebenenfalls an eine andere Station gesendet werden müssen.
\item $ \mathcal{R} < \dfrac{p-1}{p} \mid \mathcal{R}_i \mid + \dfrac{p-1}{p} \mid \mathcal{S}_i \mid $\\
Annahme: R und S sind gleichmäßig auf die p Knoten verteilt (d.h. die Partitionen sind alle gleich groß)\\
$ \mathcal{R} < \dfrac{p-1}{p} \cdot \dfrac{1}{p} \cdot \mid \mathcal{R} \mid + \dfrac{p-1}{p} \cdot \dfrac{1}{p} \cdot \mid \mathcal{S} \mid $\\
$ \Longleftrightarrow \mathcal{R} < \dfrac{p-1}{p^2-p-1} \mid \mathcal{S} \mid $
\item Alle Knoten $i\neq k$ schicken gleichzeitig $D$ Daten an Knoten $k$. Dieser empfängt mit Bandbreite $b$.\\
Die gesamte Zeit, bis Knoten $k$ alle Daten vollständig erhalten hat, ist somit $t_{vj} = \frac{(n-1)D}{b}$.
\item Die Knoten müssen jetzt insgesamt $p(n-1)=n(n-1)$ Partitionen an Knoten $k$ verschickt werden, also: $t=n\frac{(n-1)D}{b}$. Es dauert also $n$ mal länger als beim verteilten Join.\\
\item Knoten $1$ muss jetzt doppelt so viele Daten senden, die Zeit ingesamt ergibt sich dann zu $t=(n-2)\frac{D}{b} + \frac{2D}{b}$
\item Wenn $p>n$ und keine redundanten Partitionen vorhanden sind, muss jeder Knoten u.U. nicht alle seine Tupel senden, sondern es reicht, eine seiner Partitionen zu senden.\\
Wenn $p>n$ und es vorkommt, dass Partitionen redundant auf verschiedenen Knoten vorhanden sind, dann kann bei einem Join ein Knoten, der seine Partition an Knoten $k$ senden soll, aus allen Knoten mit der benötigten Partition ausgewählt werden. Z.B. kann der Knoten, der am nächsten zum Knoten $k$ liegt, ausgewählt werden, oder der Knoten, der am wenigsten ausgelastet ist. 
\end{enumerate}

\newpage
\section*{Aufgabe 2}
\begin{enumerate}[label=\alph*)]
\item $ min\_data.lp $
\begin{itemize}
\item Partition 0: Knoten 0
\item Partition 1: Knoten 0
\item Partition 2: Knoten 3
\item Partition 3: Knoten 0
\item Partition 4: Knoten 1
\item Partition 5: Knoten 0
\item Partition 6: Knoten 0
\item Partition 7: Knoten 0
\item Partition 8: Knoten 0
\item Partition 9: Knoten 0
\item Partition 10: Knoten 0
\item Partition 11: Knoten 0
\item Partition 12: Knoten 0
\item Partition 13: Knoten 0
\item Partition 14: Knoten 0
\item Partition 15: Knoten 0
\end{itemize}

\item $ min\_antwortzeit.lp $
\begin{itemize}
\item Partition 0: Knoten 3
\item Partition 1: Knoten 0
\item Partition 2: Knoten 3
\item Partition 3: Knoten 2
\item Partition 4: Knoten 1
\item Partition 5: Knoten 2
\item Partition 6: Knoten 2
\item Partition 7: Knoten 3
\item Partition 8: Knoten 0
\item Partition 9: Knoten 0
\item Partition 10: Knoten 0
\item Partition 11: Knoten 1
\item Partition 12: Knoten 1
\item Partition 13: Knoten 0
\item Partition 14: Knoten 0
\item Partition 15: Knoten 3
\end{itemize}

\item \textit{Aufgabe a}:\\
gesamte Übertragungskosten: 313\\
Übertragungszeit: 264\\
\textit{Aufgabe b}:\\
gesamte Übertragungskosten: 328\\
Übertragungszeit: 95\\
insgesamt sieht man also, dass bei Aufgabe a) die gesamten Übertragungskosten minimiert sind, während bei b) die Übertragungszeit minimiert wurde.\\
Besonders auffällig ist hierbei, dass bei nur geringer höheren Gesamtübertragungskosten die Übertragungszeit dramatisch sinkt, weshalb in diesem Fall vermutlich b) zu bevorzugen wäre.
\end{enumerate}


\end{document}


