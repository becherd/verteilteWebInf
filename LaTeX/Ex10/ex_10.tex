\documentclass[a4paper]{scrartcl}

\usepackage[utf8]{inputenc}
\usepackage{enumitem}
\usepackage{tikz}
\usepackage{amsmath}
\usepackage{amssymb}

\usepackage{anysize}
\usepackage{multirow}
\usepackage{verbatim}
\usepackage{lscape}




\marginsize{2.0cm}{2cm}{2.5cm}{2.5cm}

\author{Gruppe 6}

\title{\textbf{VerteilteWebInf Hausaufgabe 10}}
\date{\today}




\begin{document}
\maketitle

\section*{Aufgabe 1}

Die Benchmarks wurden auf einem Macbook Pro, 3 GHz Intel Core i7, 8GB Ram mit Java 7 (64 Bit) durchgeführt. Alle Zeitangaben sind in Millisekunden (ms). Gestartet wird jeder Benchmark mit einer leeren Datenbank. Beim Einfügen wird jeder Datensatz einzeln eingefügt (kein Batch-Opertion). Nachdem alle Datensätze eingefügt worden sind wird einzeln nach jedem Element gesucht (query). Anschließend wurde jedes Element einzeln wieder aus der Datenbank gelöscht.

\subsection*{1 Telefonbucheintrag} 
\begin{tabular}{lccc}
\textbf{Operation} & \textbf{HashMap}  & \textbf{MongoDB}  & \textbf{mdbm} 	\\
Insert insgesamt &  0 & 12  & - \\
Insert / Datensatz & 0 & 12 & - \\
Query insgesamt & 0 & 3 & - \\
Query / Datensatz & 0 & 3 & - \\
Delete insegesamt & 0 & 2 & - \\
Delete / Datensatz & 0 & 2 & -
\end{tabular}


\subsection*{10 Telefonbucheintrag} 
\begin{tabular}{lccc}
\textbf{Operation} & \textbf{HashMap}  & \textbf{MongoDB}  & \textbf{mdbm} 	\\
Insert insgesamt &  0 & 16  & - \\
Insert / Datensatz & 0 & 1,6 & - \\
Query insgesamt & 0 & 6 & - \\
Query / Datensatz & 0 & 0,6 & - \\
Delete insegesamt & 0 & 6 & - \\
Delete / Datensatz & 0 & 0,6 & -
\end{tabular}


\subsection*{100 Telefonbucheintrag} 
\begin{tabular}{lccc}
\textbf{Operation} & \textbf{HashMap}  & \textbf{MongoDB}  & \textbf{mdbm} 	\\
Insert insgesamt &  0 & 82  & - \\
Insert / Datensatz & 0 & 0,8 & - \\
Query insgesamt & 0 & 52 & - \\
Query / Datensatz & 0 & 0,5 & - \\
Delete insegesamt & 0 & 81 & - \\
Delete / Datensatz & 0 & 0,8 & -
\end{tabular}

\subsection*{1.000 Telefonbucheintrag} 
\begin{tabular}{lccc}
\textbf{Operation} & \textbf{HashMap}  & \textbf{MongoDB}  & \textbf{mdbm} 	\\
Insert insgesamt &  1 & 417  & - \\
Insert / Datensatz & 0 & 0,4 & - \\
Query insgesamt & 0 & 235 & - \\
Query / Datensatz & 0 & 0,2 & - \\
Delete insegesamt & 0 & 354 & - \\
Delete / Datensatz & 0 & 0,3 & -
\end{tabular}

\subsection*{100.000 Telefonbucheintrag} 
\begin{tabular}{lccc}
\textbf{Operation} & \textbf{HashMap}  & \textbf{MongoDB}  & \textbf{mdbm} 	\\
Insert insgesamt &  16 & 13.202   & - \\
Insert / Datensatz & 0 & 0,1 & - \\
Query insgesamt & 12 & 10.724  & - \\
Query / Datensatz & 0 & 0,1 & - \\
Delete insegesamt & 15 & 14.670  & - \\
Delete / Datensatz & 0 & 0,1 & -
\end{tabular}


\subsection*{1.000.000 Telefonbucheintrag} 
\begin{tabular}{lccc}
\textbf{Operation} & \textbf{HashMap}  & \textbf{MongoDB}  & \textbf{mdbm} 	\\
Insert insgesamt &  253 & 115.327 & - \\
Insert / Datensatz & 0 & 0,1 & - \\
Query insgesamt & 62  & 105.306   & - \\
Query / Datensatz & 0 & 0,1 & - \\
Delete insegesamt & 73 & 159.931   & - \\
Delete / Datensatz & 0 & 0,1 & -
\end{tabular}


\subsection*{5.000.000 Telefonbucheintrag} 
\begin{tabular}{lccc}
\textbf{Operation} & \textbf{HashMap}  & \textbf{MongoDB}  & \textbf{mdbm} 	\\
Insert insgesamt &  2544  & 651.543   & - \\
Insert / Datensatz & 0 & 0,1 & - \\
Query insgesamt & 816  & 548.228   & - \\
Query / Datensatz & 0 & 0,1 & - \\
Delete insegesamt & 73 & 841.528   & - \\
Delete / Datensatz & 0 & 0,1 & -
\end{tabular}

\section*{Aufgabe 2}
b)\\
gegeben: verschlüsselte Nachricht $C=13$,  öffentlicher Exponent $e=3$, $n=p\cdot q=15$\\

Der Klartext $M$ kann über $C^d \mod n$ berechnet werden, wobei $d$ der private Exponent des Empfängers ist.\\
Berechne $d$: Es muss gelten, dass $e\cdot d \equiv 1 \mod \Phi(n)$ mit $\Phi(n) = (p-1)(q-1)$.\\
Wir benötigen also Primzahlen $p$ und $q$, sodass $p\cdot q = n = 15$: $p=3$, $q=5$.\\
Also gilt $\Phi(n)=4\cdot2=8$.\\
Es muss also gelten: $3\cdot d\equiv1\mod8$, also wähle $d=3$.\\
Klartext $M=C^d \mod n =13^3\mod 15 = 7$.
Der Klartext der Nachricht lautet also $7$.
\end{document}


