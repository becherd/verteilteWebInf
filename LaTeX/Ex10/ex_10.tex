\documentclass[a4paper]{scrartcl}

\usepackage[utf8]{inputenc}
\usepackage{enumitem}
\usepackage{tikz}
\usepackage{amsmath}
\usepackage{amssymb}

\usepackage{anysize}
\usepackage{multirow}
\usepackage{verbatim}
\usepackage{lscape}




\marginsize{2.0cm}{2cm}{2.5cm}{2.5cm}

\author{Gruppe 6}

\title{\textbf{VerteilteWebInf Hausaufgabe 10}}
\date{\today}




\begin{document}
\maketitle


\section*{Aufgabe 2}
b)\\
gegeben: verschlüsselte Nachricht $C=13$,  öffentlicher Exponent $e=3$, $n=p\cdot q=15$\\

Der Klartext $M$ kann über $C^d \mod n$ berechnet werden, wobei $d$ der private Exponent des Empfängers ist.\\
Berechne $d$: Es muss gelten, dass $e\cdot d \equiv 1 \mod \Phi(n)$ mit $\Phi(n) = (p-1)(q-1)$.\\
Wir benötigen also Primzahlen $p$ und $q$, sodass $p\cdot q = n = 15$: $p=3$, $q=5$.\\
Also gilt $\Phi(n)=4\cdot2=8$.\\
Es muss also gelten: $3\cdot d\equiv1\mod8$, also wähle $d=3$.\\
Klartext $M=C^d \mod n =13^3\mod 15 = 7$.
Der Klartext der Nachricht lautet also $7$.
\end{document}


