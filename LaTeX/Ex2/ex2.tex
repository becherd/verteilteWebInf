\documentclass[a4paper]{article}
\usepackage[utf8]{inputenc}
\usepackage{enumitem}


\author{Gruppe 6}

\title{\textbf{VerteilteWebInf Hausaufgabe 2}}
\date{\today}
\begin{document}
\maketitle

\section*{Aufgabe 1}
\begin{enumerate}[label=\alph*)]
\item Socket-basierte Kommunikation bringt einige Nachteile für verteilte Systeme mit sich. Zunächst ist ein Socket eine Zwei-Wege-Verbindung zwischen einem Client und einem Server. Parallelität bei verteilten Anwendungen muss also explizit vom Programmierer berücksichtigt werden, da Sockets per se keine Abstraktion / Transparenz für parallele Anwendungen liefern.  
Bei der Verwendung von Datagram-Sockets welche Daten per UDP übertragen, muss (manuell) sichergestellt werden, dass die UDP Nachteile wie z.B. Packetloss ausgeglichen werden. Bei Sockets die eine TCP Verbindung aufbauen, muss der Verbindungsauf- und Abbau explizit programmiert werden. Bei verteilten Systemen muss die Verbindung zu jedem der beteiligten Server sichergestellt und überwacht werden. 

\item
\item
\end{enumerate}

\section*{Aufgabe 2}
\begin{enumerate}[label=\alph*)]
%%%%%%%%%%%%%%%%%%%%%%%%%
%%%  Somebody please verify this solution :)  %%%%
%%%%%%%%%%%%%%%%%%%%%%%%%
\item Einzelnes Tupel $t$ von A nach B übertragen: 
\[t_{A\rightarrow B} = \frac{r\cdot k_r}{b}+r\cdot l = r\left(\frac{k_r}{b} + l \right)\]
Menge von Tupeln von B nach A übertragen:
\[t_{B\rightarrow A} = \frac{s/j\cdot k_s}{b}+\frac{s}{j}\cdot l = \frac{s}{j}\left(\frac{k_s}{b} +l\right)\]
$\Rightarrow$ \[t_{ges} = t_{A\rightarrow B} + t_{B\rightarrow A} =  r\left(\frac{k_r}{b} + l \right) +  \frac{s}{j}\left(\frac{k_s}{b} +l\right)\]

\item
\end{enumerate}

\section*{Aufgabe 3}

\end{document}


